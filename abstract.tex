\chapter*{Abstract}                      % ne pas numéroter
\phantomsection\addcontentsline{toc}{chapter}{Abstract} % inclure dans TdM

\begin{otherlanguage*}{english}
Gene co-expression network analysis is a tool that entered the transcriptomics analysis toolbox 15 years ago. By studying the variation in the synchronization of gene expression, this tool can reveal new genes involved in diseases or phenotypes whose expression alone is not significantly different. It is also able to detect groups of genes, or modules, that interact preferentially and on which it is possible to carry out an extended exploration. It is therefore possible to use knowledge-driven methods such as gene enrichment or phenotypic association, or data-driven methods such as topological analysis or differential co-expression. Nevertheless, this type of analysis is currently under-exploited compared to its potential, especially in certain diseases or phenotypes where the alteration is a disorganization of the system such as aging. 

In order to facilitate the use of this method by any researcher, an R software package available on Bioconductor and named GWENA has been developed. Organized as a simplified analysis pipeline from the construction of the network to the interpretation of the modules between different conditions, it is also the only current pipeline to integrate the differential co-expression. To assist the user, it includes numerous warnings about the integrity of the data entered and the plausibility of the results. In order to limit the use of other software, it also contains a network visualization system. Finally, GWENA is a tool whose modular architecture allows it to evolve over time.

The effectiveness of GWENA has been demonstrated in a first study of human skeletal muscle aging, where a subset of genes was prioritized for the study of sarcopenia. It also allowed to clarify a network topology specific to aging and previously observed: the loss of network connectivity, or disconnection. Indeed, in parallel to the disconnection, a local reconnection located at the level of hub genes was observed thanks to GWENA. To study this topology on a large scale, the analysis was repeated on an extended set of human tissues. By cross-referencing differentially expressed modules, common aging phenomena between tissues were identified as well as tissue-specific phenomena. Topological analysis, including disconnection, of the genes included in these overlaps for two examples, a common and a specific phenomenon, in turn allowed the prioritization of genes still poorly studied or unknown in these phenomena.

Overall, the work presented in this thesis will have led to the creation of a useful tool for the community of biologists as bioinformaticians to facilitate access to a high-potential analysis in the analysis of aging and any other condition, especially those focused on the deregulation of systemic expression.

\textbf{Keywords:} co-expression, network, aging, transcriptomics, R package, Bioconductor, differential co-expression.
\end{otherlanguage*}
