\chapter*{Conclusion}         % ne pas numéroter
\phantomsection\addcontentsline{toc}{chapter}{Conclusion} % dans TdM


Après avoir présenté une vue d'ensemble des technologies de transcriptomique et leur intérêt dans la recherche en médecine moléculaire, cette thèse s'est attardée sur une présentation détaillée de l'analyse par réseaux de co-expression de gènes. Les méthodes de préparation des données, de construction du réseau, de détection des modules et leur exploitation avec ou sans connaissance a priori ont été examinées par rapport aux connaissances dans la littérature actuelle. Chaque étape demandant un niveau de maîtrise poussé en biostatistique ou théorie des graphes, il était important d'expliciter chacun des choix fait pour mieux comprendre la démarche poursuivie durant ce doctorat. 
L'adéquation de l'analyse par réseaux de co-expression de gènes pour l'étude du vieillissement a ensuite été mise en avant après une présentation concise du vieillissement, de ses manifestations cellulaire et moléculaire, ainsi que les enjeux qu'il représente.

Face à la difficulté d'emploi des outils existant sans expertise, la pénibilité de leur combinaison et de la faible pérennité d'une analyse avec les outils actuels pour réaliser une analyse par réseaux de co-expression de gènes, on avait formulé une première hypothèse qui s'interrogeait sur la faisabilité d'un outil pipeline sous la forme d'un progiciel R déposé sur Bioconductor pour répondre à ces problèmes. Grâce au développement du progiciel GWENA présenté en \hyperref[chapter:gwena]{Chapitre 1} on a démontré qu'il était possible d'avoir un outil facilitant l'analyse par réseaux de co-expression de gènes de bout en bout et qui, grâce à une architecture modulaire, était voué à s'adapter aux dernières avancées méthodologiques sur ce type d'analyse. Avec une étude de cas dédiée au vieillissement du muscle, il a également été possible de montrer la valeur ajoutée de la co-expression différentielle dans la priorisation de gènes d'une condition donnée. L'analyse de topologie a elle permit de venir préciser un phénomène déjà observé chez la souris mais pas encore chez l'homme à ce jour : une déconnexion modulaire de gènes dans le réseaux avec le vieillissement. Mieux encore, il a été mis en évidence que cette déconnexion modulaire s'accompagnait d'une reconnexion centralisée autour des gènes pivots dans le réseau.

Fort de cette capacité de GWENA à innover dans la recherche sur le vieillissement, la seconde hypothèse supposait qu'il serait à même de trouver de nouveaux gènes candidats par co-expression différentielle non pas simplement entre deux tranches d'âge, mais en plus à travers plusieurs tissus. Le \hyperref[chapter:multidim]{Chapitre 2} présente donc l'utilisation des différents outils intégrés à GWENA pour réaliser une analyse parallélisée de couples tissu et tranche d'âge. L'identification parmi les modules modérément ou non préservés de nombreux phénomènes du vieillissement tels qu'énoncés dans les marques principales du vieillissement établies par López-Otín \textit{et al.} a permit de valider la démarche de co-expression différentielle et d'aller plus en avant dans l'exploitation de ces modules. Par un recoupement des gènes contenus dans chacun des modules, des gènes impliqués dans des phénomènes du vieillissement spécifiques à quelques tissus ou à l'inverse commun à la majorité ont été détectés. Une analyse topologique faite dans un exemple de gènes spécifiques et un exemple de gènes communs a finalement retourné de nouveaux gènes pertinents dans la compréhension du vieillissement au vu des annotations dont ils bénéficiaient déjà.

Chacun de ces chapitres aura donc, en plus de montrer l'intérêt de GWENA, permis de détecter de nouveaux gènes candidats au vieillissement humain. Ceux-ci devront à l'avenir faire l'objet de validation expérimentale.

