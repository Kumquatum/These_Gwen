\chapter*{Conclusion}         % ne pas numéroter
\phantomsection\addcontentsline{toc}{chapter}{Conclusion} % dans TdM

Une thèse ou un mémoire devrait normalement se terminer par une
conclusion, placée avant les annexes, le cas échéant. Celle-ci est
traitée comme un chapitre normal, sauf qu'elle n'est pas numérotée.


% Publi utile pour les perspectives :
% https://onlinelibrary.wiley.com/doi/abs/10.1111/gbb.12106


Le vieillissement est un phénomène dont la complexité n'a d'égal que la multiplicité de ses manifestations. Tantôt origine, tantôt conséquence, les phénomènes de sa manifestations impactent de nombreux mécanismes cellulaire et moléculaires : sénescence et cellules souches, inflammation chronique de faible intensité, raccourcissement des télomères, etc. \cite{Lopez-Otin2013}. Pour distinguer de façon certaine chacun des altérations basales chez chaque individu, un nombre important d'échantillons est nécessaire. L'étude GTEx est à ce jour le plus gros regroupement de séquençage en terme d'individus/tissus non ciblé sur une pathologie. Sa réalisation permet d'étudier le vieillissement non plus dans ses manifestations les plus détaillées mais bien de rechercher les mécanismes communs à beaucoup des tissus composant le corps humain. Cependant