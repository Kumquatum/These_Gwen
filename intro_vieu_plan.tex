
\section{Les méthodes d'analyse de l'information transcriptomique pour l'étude comparative}

% La démocratisation de la quantification par puce à ADN et encore plus par RNA-seq a entraîné une explosion de la taille des jeux de données de transcriptome, ainsi que leur nombre disponible publiquement en ligne.

L'explosion de la taille et quantité de jeux de données mis à disposition publiquement engendre une demande croissante en techniques d'analyse capable de les gérer mais également de profiter de la précision de séquençage apportée. En recherche en médecine moléculaire et plus largement en biologie comparative, ce sont des méthodes d'étude des différences d'expression de gènes entre conditions qui ont été largement employées pour .









Afin d'identifier et comprendre les gènes et transcrits acteurs de différentes condition et plus spécifiquement de maladies, l'analyse transcriptomique dispose de plusieurs outils. 


Pour parvenir à déterminer la fonction physiologique des gènes, des transcrits, ou leur action dans le cadre d'une condition, plusieurs outils d'analyse sont à disposition.

% les analyses d'expression de gènes ont plusieurs outils à disposition. 



 



% Les analyses de l'expression des gènes ont pour but majeur l'identification de gènes responsables o

% \subsection{L'expression différentielle : les acteurs majeurs}
\subsection{L'analyse différentielle : les acteurs majeurs du changement}


- Méthode de capture des acteurs majeurs
- Pas d'étude du système 
- splicng alternatif
- expression differentielle
% - Des statistiques descriptives à l'expression différentielle %% Bof, j'ai pas retrouvé de stade "stats explo" 


% \subsection{La modélisation du vivant : des acteurs au système}

- De la régulation des gènes à son approximation par des modèles statistiques, en passant par la biologie des systemes qui est trop couteuse pour des organismes complexes %% mal formulé car dans le desordre (le dernier point devrait venir en 2e)
- reseaux interaction proteine proteine
- Transistion vers la modélisation par encodage de l'information dans des réseaux qui sont en fait des graph et qui sont moins couteux car probabilistes (ils ne sont pas la vérité mais une approximation)




\subsection{Les réseaux biologiques : le système vu via la théorie des graphes}

- Principe : les graphes sont une méthode de plus en plus utilisée dans la représentation du fonctionnement d'organismes puisqu'ils permettent d'avoir une vision à l'échelle du système tout entier.
%% Jolie intro à s'inspire : https://www.nature.com/articles/s41467-019-08746-5
- Différences / ressemblance entre graphe et reseau ?
- Les types de réseaux/graphes rencontrés en biologie et plus particulièrement en expression des gènes : small-world
- Les problèmes de visualisation de grands réseaux %% section layout de ce livre https://sites.fas.harvard.edu/~airoldi/pub/books/BookDraft-CsardiNepuszAiroldi2016.pdf
