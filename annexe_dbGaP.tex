\chapter{Demande d'accès dbGaP aux données protégées de GTEx}
\label{annexe:dbgap}

\section{Query title}
Muscle gene co-expression in multiple age range for sarcopenia condition exploration

\section{Research use statement}
As a multi factorial phenomena, aging is a complex condition to study. Current analysis of transcriptomic data joined with phenotypic ones have unravel few genes and environmental variables impacting it. However, aging remains not fully understood. These may come from current approaches focusing on single factors while aging is more about interaction between many of them. This is why we would like to study it through the spectrum of the co-expression networks. However, because aging is already complex by itself, we will focus on a single tissue in the beginning : muscle. Reasons are myopenia and dynapenia (known together as sarcopenia) are responsible for loss of autonomy, weak metabolic aggression resistance, and increased mortality.
By using our yet to be published pipeline developed in our lab, we aim to build gene co-expression modules and network from transcriptomic data and characterize them with external resources. This pipeline begin with a quality assessment of the data, based on the technique used to get transcriptomic information (either microarray or RNA-Seq). It then compute co-expression levels and detects modules by using the WGCNA package. Additionnal steps are then performed to fully charachterize the modules : biological enrichment, topological study, phenotipic association, differentially expressed and condition-specific gene positionning.The external resources used to it include : enrichment databases (GO, KEGG, etc.), phenotypic information (exact age, condition of death, ethnicity), and age related databases (Digital aging atlas, Aging map, etc.). The use of these complementary information represents no additional risk to participants since only summarized gene information we'll be used in the form of modules. No other raw transcriptomics data will be combined to the GTEx data, but only a comparison of the modules built on each datasets in order to study the reproductibility of our modules, and the comparison with modules built on sarcopenia dedicated datasets.
Phenotipyc information is the main reason for our current demand regarding protected datasets because we think precise age and cause of death will impact the aging expression. 
No collaboration with other institution is planned.
Finnally, this methodological work will advance the understanding of the genetic bases of the aging processes occurring in the muscle.


\section{Non-technical summary}
Aging affect every one of us. It is defined as the progressive degradation of biological functions inside the body. In the muscle, this results in a decreasing in muscle density and strength called sarcopenia. They are responsible for mobility difficulties, therefore autonomy, and a deficit in body protection against aggression, sometimes leading to death. With this risks at stake, it is understandable that a better understanding of aging processes in muscle is important for public health. Few single genes linked to it have already been discovered, however we still don't fully understand the mechanisms occurring and how to limit them. Gene expression is a witness of the changes in the body mechanisms. By studying the expression of the genes and the coordination between this levels of expression, therefore patterns of expression, we aim to detect news genes involved in muscle aging. Do to so, we regroup the most coordinated genes in entities called module and try to unravel the biological interactions which link them. Because some of this genes are already involved into other biological interactions, we can have an insight on the regulations that take place in this case by extrapolating information. These, will lead to the discovery of new genes associated with sarcopenia.

\hfill \break
\textit{Réalisé le 30/09/2019}

\textcolor[RGB]{220,220,220}{\rule{\linewidth}{0.2pt}}

\section{Access update : Research Progress}
Current analysis of the GTEx data focused on the skeletal muscle samples as we are studying sarcopenia and other aging impact on muscle. The RNAseq data was then split in two opposite age range (young and old) to emphase and capture the differences in aging through our newly developped differential gene co-expression network pipeline GWENA (\url{https://www.bioconductor.org/packages/release/bioc/html/GWENA.html}). 
A first analysis solely on the modules (gene groups) detected by GWENA in the young condition. A selection of interesting modules for muscle activity was achieved through a combinaison of moduels phenotypic association and gene sets enrichment. Inside one promissing module, hub genes connected in the network to genes involved in muscle activity allowed the identification of genes with strong evidence of contribution to muscle developpemnt and growth.
A second analysis used the full potential of the differential co-expression analysis offered by GWENA to find specific co-expression modules to aging. Important topological variation lead us to focus on one module. Further investigation is currently underway to characterise the origin of the specificity and this variations. 

\hfill \break
\textit{Réalisé le 01/12/2020}
