\chapter*{Remerciements}         % ne pas numéroter
\phantomsection\addcontentsline{toc}{chapter}{Remerciements} % inclure dans TdM

La recherche n'est pas un processus en ligne droite, ce n'est pas un processus où celui qui est en avance va systématiquement trouver la nouvelle réponse à une grande question. C'est une avancée collective ou la moindre découverte n'est en aucun cas l'œuvre d'une seule personne. Il aura au préalable fallut accumuler la connaissance sur laquelle repose ladite découverte, il aura fallu le soutien des proches qu'on oublie trop souvent pour leur contribution certes indirecte mais essentielle. La recherche c'est une avancée à tâtons qui devrait donner sa chance à chaque branche de l'arbre qu'est la recherche car il est bien impossible de prédire exactement quelle sera la prochaine à donner un fruit. Qui plus est un fruit intéressant vis-à-vis de notre utilitarisme contemporain.

Pour pouvoir profiter des rares moments de joie suite à un résultat, une découverte, qui nous font aimer ce métier de chercheur, il faut donc dans la recherche faire preuve d'abnégation ou de résilience. Je ne saurais dire laquelle de ces deux notions de persévérance évoquées sied le plus à la description de ces 4 ans de doctorat. Ce que je sais, c'est qu'il n'aurait pas été possible de les tenir sans la contribution de certains et sans le soutien inconditionnels d'autres. Cette page a donc pour but de remercier tous ces gens dans un ordre indépendamment de leur degré de contribution à cette thèse.

Je remercie donc mon directeur de thèse Arnaud Droit pour m'avoir accueillie dans son laboratoire.

Je remercie les collaborateurs de L'Oréal dans le cadre de la Chaire de Recherche en Biologie numérique pour m'avoir donné l'opportunité d'y contribuer.

Merci aux membres du jury, la Dre. Francine Durocher, le Dr. Simon Hardy, le Dr. Yohan Bossé, la Dre. Sarah Gagliano Taliun, pour avoir accepté de siéger et évaluer mes travaux.

Merci à Marie Pier Scott-Boyer, superviseure de mes débuts de doctorat puis collègue source de nombreuses discussions scientifiques. À nos désaccords de Recherche sincèrement bénéfiques pour ma formation, même si ce n'était pas de la façon que je l'aurai pensée.

Merci à Julien Prunier, professionnel de recherche qui a su jouer un avocat du diable parfois plus vrai que nature à grands renforts de mauvaise foi assumée pour me forcer à apporter de la nuance dans mon entêtement constant. 

Merci à Charles Joly-Beauparlant, professionnel de recherche et véritable rocher immuable dans la tempête des projets s'accumulant et toujours ravi de donner un coup de main. Tu resteras le grand R master à mes yeux et ce fut un plaisir d'échanger cuisine avec toi.

Merci à Éric Fournier, professionnel de recherche à qui je dois le nom de mon package Bioconductor GWENA suite à une blague en réunion de labo. On va me penser narcissique à tort, mais comme tu l'as dit "Ça fonctionne trop bien et sera peut-être qu'une fois dans ta vie".

Merci aux autres membres de l'ADLab qui ont contribué à cette expérience qu'est le doctorat d'une façon ou d'une autre. Nos parties de babyfoot me manqueront. Et merci à ces anciens du labo qui ont en leur temps contribué à cette thèse. Merci donc à Maxime Vallée, professionnel de recherche pour son honnêté dans nos échanges sur la vie professionnelle et québécoise. Merci à notre ancien homme de l'ombre, véritable Batman de l'administration système, Adrien Dessmond.

Merci à Alexandra Elbakyan sans qui nombreuses seraient les publications qui me seraient restées hors de portée.

Des mercis en masse à toutes ces rencontres en terre canadienne qu'une thèse en un autre lieu n'auraient su me donner. : Marine, Camille, Charlie, Clément, Thomas, Xavier, Antoine. Votre bonne humeur et nos sorties découvertes de la culture/nature allez me manquer.

À cette communauté qu'est celle de la vulgarisation scientifique et notamment celle du Vortex, un merci sincère. Partager la science est un plaisir tant pour vous que pour moi alors on va continuer comme ça. Un merci supplémentaire à Mel, Mallou, Piloy, Kimist, Penta, Flax, Clem, Pha, Tofu, Pollo, Viper, Ilwa, Jerry, Bubble, Egz, Hibou, Apo, Oldu et toutes ces autres personnes que je ne pourrai pas citer sans risquer de faire des remerciements plus longs que ma thèse elle-même.

Pour cette confiance que je ne m'accorde jamais mais qu'ils ont su me donner, merci à mes coéquipiers administrateurs de Bioinfo-fr.net Isabelle et Yoann de m'avoir recrutée et donné l'occasion de faire mes preuves en publication et gestion d'édition. Un merci un peu long également aux membres de son canal IRC car ils sont nombreux à avoir enrichi scientifiquement et personnellement mes réflexions durant ce long doctorat. Un merci en ordre alphabétique, pour qu'ils ne trouvent pas une n-ième raison de troller, à Aestra, aurelbzh, azerin, Billbis, bjonnh, Chopopope, eorn, lhtd, Lins`, maxulysse, MoUsSoR, Natir, Nedgang, neolem, noctisLab, Norore, schneu, slybzh, YaknotiS, ZaZo0o.

Un merci tout particulier à ma famille pour son soutien de toujours et nos skypes dominicaux. À Isabelle et Florent, mes parents, car on a beau être adulte, partir au bout du monde, crier haut et fort son indépendance, on reste toujours l'enfant de ses parents. Je suis fière d'être votre fille. À Alexandre et Elzia, parce qu'à 3 enfants terribles on a toujours su se soutenir à notre façon. À mes grands-parents, Georges et Geneviève, véritables forces de la nature et source d'un profond respect.

Vous l'avez déjà entendu mille fois mais une fois de plus n'est pas de trop : merci à Romain, Jacques, Gabriel, Lauriane, Camille, Margaux, Maxime, compagnons de jeux vidéos, de société, ou de rôle qui malgré les milliers de kilomètres nous séparant ont répondu présent à l'appel et ont permis de parler librement, de décompresser. À nos rires, nos sessions chant et notre sel qui ne saurait égaler celui de nos adversaires face à nos âneries.

Pour avoir répondu à l'appel nuit et jour malgré le décalage horaire, avoir été un soutien indéfectible en croyant bien plus à ma réussite que moi-même, avoir ramené de la raison où j'en manquais, avoir épanché mes douleurs de maladies chroniques physique et mentale que le doctorat n'aura clairement pas amélioré, avoir contribué au soin de cette dernière, avoir menacé tour à tour de prendre l'avion pour venir me botter le cul, un millier de mercis ne suffirait pas à Léopold et Amandine. Un câlin supplémentaire à Léopold qui pourra presque prétendre à l'HDR avec tous ces points scientifiques qu'on aura faits, et une promesse de session cuisine et canapé à ma toupine.

Un merci plus que particulier à Raijich (Régis) et Audrey, camarades doctorants devenus des amis chers à mon cœur. Je n'oublierai jamais à quel point je leur dois d'avoir tenu jusqu'au bout. Vous avez été là à chaque fois que j'en avais besoin consciemment ou non et on a pu progresser ensemble en se serrant les coudes. Je souhaite de tout mon cœur qu'on poursuive nos sessions détox régulières même après ce doctorat. Merci Audrey pour les rires, le mentorat, les sorties avec des diamants plein les yeux ou couronnée d'un poster trophée, la conversion au sous-estimé legging. Merci à Régis pour ce partage si subtile que sont l'humour absurde et la culture internet, pour ces papotages informatique au sens large ou cuisine, pour les grandes discussions en pleine crise existentielle tard (beaucoup trop tard) dans le labo, pour la reprise de l'escalade en bloc au grand dam de mes mains à présent couvertes de corne. À vous deux, à nos lavages de tasse à thé !

Enfin, à cette personne connue il y a bientôt 6 ans mais que j'aurai attendu 4 ans de plus avant de me rendre compte qu'il était la moitié que j'attendais pour me sentir complète, merci Clément. Ton soutien sans réserve pour m'aider à tenir jusqu'à la fin, ta tolérance face à mes râleries et excès, ton réconfort lors de mes doutes les plus viscéraux, font de toi l'homme que j'ai envie de chérir du plus profond de mon être. Je saurais te rendre la pareille pour ta (fausse) thèse ;D. À notre aventure à deux que j'espère la plus longue possible, nous Sain(t) Clément et le Graoully.