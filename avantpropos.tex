\chapter*{Avant-propos}         % ne pas numéroter
\phantomsection\addcontentsline{toc}{chapter}{Avant-propos} % inclure dans TdM

% L’avant-propos contient les renseignements sur:
% - l’état de publication des articles intégrés (dates de soumission, d'acceptation ou de publication)
% - les modifications entre la version intégrée de l’article et sa version publiée, s’il y a lieu 
% - votre statut d’auteur (principal ou non)
% - votre rôle exact dans la préparation de chaque article
% - les coauteurs de chaque article

\section{Projets principaux}

Cette thèse est réalisée avec l'insertion d’articles écrits durant mon doctorat. Elle présente l’état de mes travaux dont le but principal était le développement d’outils et méthodes pour la détection de gènes candidats au vieillissement humain par l'utilisation de réseaux de co-expression de gènes. Chaque chapitre est donc constitué d'un article publié ou visant à l'être.

Les articles insérés sont les suivants :
\begin{itemize}
    \item \textit{GWENA: gene co-expression networks analysis and extended modules characterization in a single Bioconductor package}, publié dans la revue \textit{BMC Bioinformatics} le 25 mai 2021.
    \item \textit{Analyse trans-tissus par réseau de co-expression de gènes pour la détection de fonctions physiologiques communes et spécifiques au vieillissement}, article en préparation.
\end{itemize}

Auteurs impliqués :
\begin{itemize}
    \item Gwenaëlle G. Lemoine, Département de médecine moléculaire, Faculté de médecine, Université Laval, 2325 rue de l’Université, Québec G1V 0A6, Canada. 
    \item Marie Pier Scott-Boyer, Centre de recherche du CHU de Québec‑Université Laval, 2705 boulevard Laurier Québec, Québec G1V 4G2, Canada.
    \item Bathilde Ambroise, L’Oréal Research and Innovation, 15 rue Pierre Dreyfus, 92110 Clichy, France.
    \item Olivier Périn, L’Oréal Research and Innovation, 15 rue Pierre Dreyfus, 92110 Clichy, France.
    \item Arnaud Droit, Département de médecine moléculaire, Faculté de médecine, Université Laval, 2325 rue de l’Université, Québec G1V 0A6, Canada. 
\end{itemize}

\section{Contribution à l'article "GWENA"}

Je suis responsable de la conception, développement et maintenance de l'outil GWENA ainsi que de l'écriture de l'article. Concernant la réalisation de l'analyse pour le cas d'utilisation sur le muscle, je suis responsable du traitement des données ainsi que de leur analyse. Le choix de la méthodologie fut un travail conjoint de Marie Pier Scott-Boyer qui a également supervisé le projet. Elle a également avec Olivier Périn, Bathilde Ambroise et Arnaud Droit participé à la relecture de l'article. L'intégralité de l'article a été validé par tous les auteurs. Arnaud Droit s'est également chargé de la recherche de financement.

\section{Modification à l'article "GWENA"}
Des erreurs dans le texte ne modifiant pas les conclusions de l'article ont été trouvé à la relecture par le jury. Celles-ci ont été corrigées et sont listée ci-dessous :
\begin{itemize}
    \item Chapitre 1, Figure \ref{fig:fig_pipeline_schema} 2.1, points \textcircled{\small{1}} et \textcircled{\small{2}} : les colonnes et lignes du tableau ont été transposés pour correspondre au texte "a table with genes as columns and samples as rows"
    \item Chapitre 1, Section \textit{Single condition modules analysis} : une précision disparue lors des relectures avant publication et concernant la sélection des modules a été remise : "Modules 19, 21 and 25 were the top 3 enriched for terms related to muscle function" changé pour "Modules 19, 21 and 25 were the top 3 enriched for terms related to muscle function also associated with a phenotype impacting muscle"
    \item Chapitre 1, Section \textit{Single condition modules analysis}, Figure \ref{fig:fig_resume_visu}, point C de la légende : correction du module 10 en module 19
    \item Chapitre 1, Section \textit{Single condition modules analysis}, Figure \ref{fig:fig_resume_visu} : inversion des textes des points D et E pour correspondre au bon panel de la figure
    \item Chapitre 1, Section \textit{Background}, une référence oubliée à la base de donnée GTEx a été rajoutée
\end{itemize}

\section{Contribution à l'article d'analyse trans-tissus du vieillissement via GWENA}

Je suis responsable de la conception du projet, du traitement et de l'analyse des données, de l'interprétation des résultats, et de la rédaction de l'article. Marie Pier Scott-Boyer a assisté dans la consolidation de la méthodologie et sa validation. Arnaud Droit s'est chargé de la recherche de financement.


\section{Projets annexes}

Durant mon doctorat, j'ai également pu m'investir dans différents projets scientifiques :
\begin{itemize}
    \item \textit{Weighted gene co-expression network analysis identifies inflammaging biomarkers in aged skin of humans in vivo}. Projet de ré-analyse des données de transcriptomique de Kuehne et al. 2017 par le biais de GWENA. Il a permis de mettre en évidence des gènes impliqués dans le phénomène d'inflammation chronique de faible intensité dans des biopsies d'épiderme. Par respect envers la clause de confidentialité de la chaire de recherche et d'innovation L'Oréal en biologie numérique, ces travaux n'ont pas été soumis à publication.
    \item Étude à travers de multiples points temporels sur 28 jours de la reconstruction épidermique basée sur un modèle d'épiderme \textit{in vitro} provenant de biopsies de circoncisions. Travaux effectués pour la chaire de recherche et d'innovation L'Oréal en biologie numérique et confidentiels.
\end{itemize}

D'autre travaux scientifiques non académiques ont également été réalisés et sont visibles en Annexe \ref{annexe_projets_annexe}.



\section{Financements}

Les travaux présentés dans cette thèse ont étés soutenus par la Chaire de recherche et d'innovation L'Oréal en biologie numérique.

\section{Notes}

\begin{itemize}
    \item L'intégralité des figures a été réalisée par mes soins et est sous licence CC-BY-NC sauf mention contraire ou citation d'une figure d'une publication.
    \item L'article situé en \hyperref[chapter:gwena]{Chapitre 1} est publié dans BMC Bioinformatics sous la licence CC-BY.
    \item Pour plus d'information sur les licences Creative Commons : \url{https://creativecommons.org/about/cclicenses/}.
    \item Toute figure ou table reprise d'un article est cité et traduite lorsque les licences du journal le permettent.
    \item Pour des raisons de cohérence avec l'article rédigé en anglais, de cohérence avec la littérature scientifique rédigée majoritairement en anglais, les acronymes de cette thèse utilisés dans la recherche et rarement trouvés dans le langage commun seront en anglais après avoir été explicité en français lors de leur première utilisation.
\end{itemize}