% \chapter*{Glossaire}                                        % enlève la numérotation
\phantomsection\addcontentsline{toc}{chapter}{Glossaire}    % inclus le glossaire dans la table des matières

\printglossary[nonumberlist]
% \glsaddall

\footnotetext{Les définitions précédées d'un [W] sont issues de Wikipédia}

%% MODELE

%\newglossaryentry{}
%{
%	name={},
%	description={}, 
%	plural={}
%}



\newglossaryentry{}
{
	name={},
	description={}, 
	plural={}
}

\newglossaryentry{transcrit}
{
	name={transcrit},
	description={version d'ARN messager issue d'un gène}, 
	plural={transcrits}
}

\newglossaryentry{aging_hallmarks}
{
	name={marque principale},
	description={représentation des causes et conséquence du vieillissement en neuf marques réalisée par López-Otín \cite{Lopez-Otin2013}}, 
	plural={marques principales}
}

\newglossaryentry{plus_court_chemin}
{
	name={plus court chemin},
	description={suite de liens d'un nœud à un autre de longueur la plus petite possible}, 
	plural={plus courts chemins}
}

\newglossaryentry{connectivite}
{
	name={connectivité},
	description={somme des valeurs des liens d'un nœud. Différente définitions existent dépendant du contexte, mais c'est celle-ci qui sera entendue dans cette thèse}, 
	plural={connectivités}
}

\newglossaryentry{graphe}
{
	name={graphe},
	description={modèle abstrait mathématique fait de sommets reliés par des arrêtes pour conceptualiser des intéractions}, 
	plural={graphe}
}

\newglossaryentry{reseau}
{
	name={réseau},
	description={ensemble interconnecté de nœud par des liens ou graphe qui représentent l'organisation d'un phénomène.}, 
	plural={réseaux}
}

\newglossaryentry{gene_propre}
{
	name={gène propre},
	description={première composante d'une analyse par composantes principales sur un module de gènes}, 
	plural={gènes propres}
}

\newglossaryentry{voisin}
{
	name={voisin},
	description={nœuds reliés à un nœud considéré. On peut parler à voisins à une distance x où x représente la distance en terme de liens qu'on prendra en plus en considération}, 
	plural={voisins}
}

\newglossaryentry{degre}
{
	name={degré},
	description={nombre de liens (ou arêtes) reliant un nœud (ou sommet), avec les boucles comptées deux fois}, 
	plural={degrés}
}

\newglossaryentry{organe}
{
	name={organe},
	description={[W] groupe de tissus collaborant à une même fonction physiologique}, 
	plural={organes}
}

\newglossaryentry{type_cellulaire}
{
	name={type cellulaire},
	description={spécialisation d'une cellule au cours de sa différenciation pour la réalisation d'une fonction biologique spécifique}, 
	plural={types cellulaires}
}

\newglossaryentry{tissu}
{
	name={tissu},
	description={[W] ensemble fonctionnel de cellules au type cellulaire différent regroupées en amas, réseau ou faisceau}, 
	plural={tissus}
}

\newglossaryentry{condition}
{
	name={condition},
	description={contexte pratique dont l'existence est nécessaire pour l'observation dans un échantillon d'un phénomène. Ce terme vise dans cette thèse à englober "maladie" et "phénotype" sous un même mot}, 
	plural={conditions}
}

\newglossaryentry{phenotype}
{
	name={phénotype},
	description={ensemble des traits observables de l'échelle moléculaire à macroscopique d'un organisme}, 
	plural={phénotypes}
}

\newglossaryentry{transcriptomique}
{
	name={transcriptomique},
	description={[W] étude de l'ensemble des ARN messagers produits lors du processus de transcription d'un génome}, 
	plural={transcriptomiques}
}

\newglossaryentry{genexpr}
{
	name={expression de gènes},
	description={[W] ensemble des processus biochimiques par lesquels l'information stockée dans un gène est lue pour aboutir à la fabrication d'ARN ou transcrit}, 
	plural={expression des gènes}
}

\newglossaryentry{organisme}
{
	name={organisme},
	description={[W] ensemble des organes d’un être vivant et, par métonymie, l'être vivant lui-même}, 
	plural={organismes}
}

\newglossaryentry{transcriptome}
{
	name={transcriptome},
	description={ensemble des ARN transcrits depuis l'ADN chez un organisme donné}, 
	plural={transcriptome}
}

\newglossaryentry{longevite}
{
	name={longévité},
	description={durée de vie potentielle d'un organisme}
}

\newglossaryentry{module}
{
	name={module},
	description={groupe de gènes co-exprimés au sein d'un réseau}, 
	plural={modules}
}

\newglossaryentry{mecanisme}
{
	name={m\'{e}canisme},
	description={ensemble des acteurs cellulaires et moléculaires intervenant dans la réalisation d'une manifestation d'un phénomène \cite{Bechtel2013}}, 
	plural={m\'{e}canismes}
}

\newglossaryentry{fonction_biologique}
{
	name={fonction biologique},
	description={terme ambigu pouvant désigner de nombreux concepts en biologie. Aussi dans cette thèse on s'appliquera à préciser au maximum son sens à l'aide du modèle de Pittsburg \cite{Keeling2019Nov}:
	\begin{description}
	    \item \textit{Implications évolutives} : l'influence de l'objet sur la dynamique de la population au cours de générations successives, telle qu'elle est rendue possible par ses implications physiologiques et leur interaction avec les pressions environnementales.
	    \item \textit{Implications physiologiques} : l'implication de l'objet dans les processus biologiques, telle qu'elle est rendue possible par un ensemble de capacités, d'interactions et de modèles d'expression, indépendamment des considérations transgénérationnelles.
	    \item \textit{Interactions} : les contacts physiques, directs ou indirects, entre l'objet étudié et les autres composants d'un système, y compris les contacts qui servent de médiateurs aux transformations chimiques.
	    \item \textit{Capacités} : les propriétés physiques intrinsèques de l'objet étudié ; la nécessité du comportement de l'objet compte tenu de son environnement (par exemple, les contraintes structurelles)
	    \item \textit{Expression} : la présence ou la quantité de l'objet étudié (objet ARN ou protéine), ou la présence ou la quantité de ses produits de transcription ou de traduction (objet ADN).
	    \item \textit{Vague} : on n'a pas trouvé de preuves suffisantes pour déduire une ou plusieurs significations de la fonction dans ce modèle, ni pour dériver une nouvelle signification.
	\end{description}
	}, 
	plural={fonctions biologiques}
}

\newglossaryentry{scalefree}
{
	name={invariance d'échelle},
	description={[\textit{scale free network}] (mathématiques) réseau dont les degrés suivent une loi de puissance. C'est à dire un réseau où la proportion de nœuds de degré $k$ est proportionnelle à $k^{-\gamma}$ pour $k$ grand, où $\gamma$ est un paramètre}
}


