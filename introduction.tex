\setcounter{chapter}{1}         % permet de débuter l'intro à 1. au lieu de 0.
\chapter*{Introduction}         % enlève la numérotation
\phantomsection\addcontentsline{toc}{chapter}{Introduction} % inclus l'intro dans la table des matières

Une thèse ou un mémoire devrait normalement débuter par une
introduction. Celle-ci est traitée comme un chapitre normal, sauf
qu'elle n'est pas numérotée.

\section{La complexité du vivant}
\begin{itemize}
\item "La cellule constitue l'unité de base constituant tout être vivant. Son bon fonctionnement est assuré par l'encodage de toute l'information nécessaire dans la molécule d'ADN présente en chacune."
\item Intro sur les différents niveaux d'encodage et traduction de l'information (de ADN à proteine en passant par ARNm)
\item L'ARNm comme témoin des activités et processus cellulaires
\item La variabilité de l'expression, d'un organisme à un autre, d'un contexte à un autre (maladie, age), d'un instant t à t+1 (quelques secondes, matin/soir, lendemain, etc.)
\end{itemize}
%% Un mot sur les SNPs ? Ou inutile

\subsection{L'étude des perturbation du vivant pour la résolution de conditions cliniques}

\section{Les technologies de séquençage de l'expression des gènes}
\begin{itemize}
\item Avant propos sur le développement des technologies de séquençage avec la quantification d'ARNm par rt-qPCR, le séquençage par gel, etc.
\item Transition vers les technologies de séquençage nouvelle génération
\end{itemize}

\subsection{Microarray}
\begin{itemize}
\item Principe
\item Utilisation avec des exemple
\item Propriétés mathématiques et techniques
\begin{itemize}
    \item Distribution
    \item Normalisation : Normalization is a process designed to identify and correct
technical biases. 2 types of norm : Between and within normalization (cf. formation marie laure meme si c'est pour le RNA-seq). Pourquoi normalisation log2 : le log pour passer à une échelle symétrique autour de 0, le 2 car c'est + facile à interpréter, chaque fois qu'on augmente le ratio Ti de 1, on double la up regulation  %%(https://www.researchgate.net/post/Why_do_we_usually_use_Log2_when_normalizing_the_expression_of_genes et https://www.nature.com/articles/ng1032z)
    \item Contrôle qualité
    \item Filtration
\end{itemize}
\item Explication du déclin mais avec contraste sur son utilité quand pas besoin de whole transcriptome
\end{itemize}

\subsection{RNA-Seq}
\begin{itemize}
\item Principe
\item Un mot sur le fait qu'on applique pas les mêmes méthodes de normalisation car la nature du signal n'est pas le même : une fluorescence pour le microarray, et un comptage dans le cas du RNA-seq
\item Explication de l'essor du RNA-seq (couts, precision, etc.)
\item Propriétés mathématiques et techniques : distribution, normalisation (taille de librairie (= profondeur de sequencage), contenu en GC, taille des gènes, composition de la population d'ARN de chaque condition), contrôle qualité, filtration  (https://www.biostars.org/p/349881/)
\end{itemize}

\section{Le traitement de l'information biologique}
\subsection{L'expression différentielle}
\begin{itemize}
  \item
% \item Des statistiques descriptives à l'expression différentielle %% Bof, j'ai pas retrouvé de stade "stats explo" 
\end{itemize}

\subsection{La modélisation du vivant}
\begin{itemize}
\item De la régulation des gènes à son approximation par des modèles statistiques, en passant par la biologie des systemes qui est trop couteuse pour des organismes complexes %% mal formulé car dans le desordre (le dernier point devrait venir en 2e)
\item Transistion vers la modélisation par encodage de l'information dans des réseaux qui sont en fait des graph et qui sont moins couteux car probabilistes (ils ne sont pas la vérité mais une approximation)
\end{itemize}

\subsection{Théorie des graphes}
\begin{itemize}
\item Principe 
\item Différences / ressemblance entre graphe et reseau ?
\item Les types de réseaus/graphes rencontrés en biologie et plus particulièrement en expression des gènes : small-world
\end{itemize}

\section{Les réseaux de  co-expression}
\begin{itemize}
\item Principe / pipeline 
\item Les différents scores de similarité : Pearson, Spearman, bicor, mutual information
\item La pondération des scores 
\end{itemize}


\section*{Phrases utiles à retravailler et intégrer}

\begin{itemize}
\item "Studies have shown that each gene is estimated on average to interact with four to eight other genes1 and to be involved in 10 biological functions" [10.1038/s41598-017-18705-z]
\item "A very clear partition of different biological networks is provided by Christensen et al. [1], who separated these networks into five main categories as follows: [metabolic networks, signal transduction networks,transcriptional regulatory networks, protein-protein networks, functional gene networks]" [10.1093/bfgp/elt003]
%% Bouquin à la coloc de Clément à Sète, pas retrouvé sur le net jusque là 
\item "Le vieillissement est un continuum conduisant une personne en bonne santé à une réduction de sa réserve fonctionnelle, puis de sa capacité fonctionnelle et de sa qualité de vie. CEs différents aspects ne décrivent pas un chemin linéaire ou ordonné." [Patient agé : particularités de la consultation, Gilles Berrut]
\item "It has been reported that nearly one in four studies uses public data to address a biological problem without generating new raw data (Rung and Brazma, 2013)." [10.3389/fpls.2016.00444]
\end{itemize}


%% Points que je veux aborder :
%%  - Definition des reseau arrete / noeud
%%  - Definition des modules par l'effet de modularité
%%  - Ce à quoi ils peuvent servir : raccrochage de fonction a certains genes, detection de pathways, detection de reseaux de regulation



%%    • Weighted aspect [trop general, aura plutot sa place dans la thèse]
%% But because of the multi-functionality aspect of each gene, such GCN using only binary state between genes (1 = correlated, 0 = not correlated) lead to information loss\cite{Langfelder2008}. Therefore, a new class of GCN have been developed: weighted gene co-expression networks. One of the most famous implementation is the R package WGCNA\cite{Langfelder2008}, but one can also mention the recent wTO\cite{Gysi2018} package which consider both positive and negative correlation for GCN building. Instead of a simple pairwise correlation, these packages weight the similarity score by calculating a factor taking into account other properties of the network. In the case of WGCNA, it specifies an adjacency score which raise the similarity to a power, which will increase the strength of strong similarities while keeping low week ones. In the case of wTO, it determine an average accounting for all common neighbors of a node.

%%    • Topological aspect
%%    • Co-expression properties (Scale-free-network entre autres ?)