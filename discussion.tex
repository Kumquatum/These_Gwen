\setcounter{chapter}{5}
\setcounter{section}{0}
\setcounter{figure}{0}   
\chapter*{Discussion}         % ne pas numéroter
\phantomsection\addcontentsline{toc}{chapter}{Discussion} % dans TdM

\section{L'apport de mes travaux sur l'étude du vieillissement}

\begin{itemize}
    \item La création d'un outil de type pipeline avec comparison et analyse intégrée
    \item La démonstration de son application chez le muscle et le ciblage de gènes potentiellements liés au vieillissement
    \item whatever secodn article
    \item Les études de quantification de l'expression c'est bien, mais grâce au RNA-seq on peut aller explorer d'autres pan du transcriptome et potentiellement faire des recoupements avec des phénomènes/mécanismes observés via quantification de l'expression
\end{itemize}

\section{limitations}
\begin{itemize}
    \item Les réseaux de co-expression comme moyen d'étude mais de nouvelles méthodes moins bruitée se mettent en place %% https://www.nature.com/articles/s42255-020-00304-4 et https://www.nature.com/articles/s42255-020-00295-2 (pas un article mais un News and Views qui resume)
    \item Ne permettent pas directement de connaitre le sens de régulation, il faut d'autres analyses pour inférer le sens de la relation dans un réseau de co-expression. Avec ça vient les soucis d'inférence évoqués par marie laure : co-regulation, manque de puissance stat..
\end{itemize}



\section{Perspectives}

- la potentielle application dans l'étude du cancer ou dans une étude comparative concernant la perte de connectivité observée là bas également 10.1371/journal.pone.0087075
- l'utilisation dans des populations d'ethinies plus minoritaires, malheureusement les données et bdd sont faibles par rapport à la puissance stat requise
- Il existe d'autres types de réseaux complémentaires "A very clear partition of different biological networks is provided by Christensen et al. [1], who separated these networks into five main categories as follows :" cf https://academic.oup.com/bfg/article-lookup/doi/10.1093/bfgp/elt003
- approches multi-omiques avec intégration de réseaux protein-protein % <- ptet plus dans mise en contexte des travaux