\chapter*{Résumé}                      % ne pas numéroter
\phantomsection\addcontentsline{toc}{chapter}{Résumé} % inclure dans TdM

\begin{otherlanguage*}{french}
L'analyse par réseau de co-expression de gènes est un outil entré il y a 15 ans dans l'ensemble des outils disponibles pour l'analyse transcriptomique. En étudiant la variation de synchronisation de l'expression des gènes, cet outil permet de révéler de nouveaux gènes impliqués dans des maladies ou phénotypes dont l'expression seule n'est pas significativement différente. Il est également capable de détecter des groupes de gènes, ou modules, interagissant préférentiellement et sur lesquels il est possible d'effectuer une exploration étendue. Il est ainsi possible d'utiliser des méthodes avec injection de connaissance préalable comme l'enrichissement de gènes ou l'association phénotypique, ou des méthodes guidées par les données comme l'analyse topologique ou la co-expression différentielle. Pourtant, ce type d'analyse reste sous exploitée actuellement par rapport à son potentiel, et notamment dans certaines maladies ou phénotypes où l'altération est une désorganisation du système comme le vieillissement. 

Afin de faciliter à tout chercheur l'emploi de cette méthode, un progiciel R disponible sur Bioconductor et nommé GWENA a été développé. Organisé comme un pipeline d'analyse simplifié et allant de la construction du réseau jusqu'à l'aide à l'interprétation des modules entre différentes conditions, c'est également le seul pipeline actuel à intégrer la co-expression différentielle. Pour assister l'utilisateur, il comprend de nombreux avertissements sur l'intégrité des données rentrées et sur la plausibilité des résultats. Afin d'éviter de devoir recourir à d'autres logiciels, il contient également un système de visualisation des réseaux. Enfin, GWENA est un outil dont l'architecture modulaire lui permettra d'évoluer avec le temps.

L'efficacité de GWENA a été démontrée dans une première étude du vieillissement du muscle squelettique humain où un sous ensemble de gènes a été priorisé pour l'étude de la sarcopénie. Il a également permis de préciser une topologie du réseau spécifique du vieillissement et observée auparavant : la perte de connectivité du réseau, ou déconnexion. En effet, parallèlement à la déconnexion, il a été constaté grâce à GWENA une reconnexion locale située au niveau des gènes pivots. Pour étudier cette topologie à large échelle, l'analyse a été répétée sur un ensemble élargi de tissus humains. Par un recoupement des modules différentiellement exprimés, des phénomènes communs du vieillissement entre tissus sont apparus ainsi que des phénomènes spécifiques à certains tissus. L'analyse topologique, notamment de la déconnexion, des gènes inclus dans ces recoupements pour deux exemples, un phénomène commun et un phénomène spécifique, a à son tour permis la priorisation de gènes encore mal étudiés ou inconnus dans ces phénomènes.

En finalité, les travaux présentés au cours de cette thèse auront amené à la création d'un outil utile à la communauté de biologistes comme bio-informaticiens pour faciliter l'accès à une analyse à a haut potentiel dans l'analyse du vieillissement et toute autre condition, notamment celles axées sur la dérégulation de l'expression systémique.

\textbf{Mots-clefs :} co-expression, réseau, vieillissement, transcriptomique, progiciel R, Bioconductor, co-expression différentielle.
\end{otherlanguage*}
