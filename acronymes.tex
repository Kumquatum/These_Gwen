% \chapter*{Acronymes}                      % ne pas numéroter
% \phantomsection\addcontentsline{toc}{chapter}{Glossaire} % inclure dans TdM

% \newglossaryentry{esptopo}
% {
% 	name={espace topologique},
% 	description={(mathématiques) ensemble muni d'une structure très générale (la topologie), qui permet de définir la notion de voisinage d'un point. Cette structure (la topologie) offre le langage pour définir les notions de continuité et de limite. Deux définitions équivalentes sont souvent données : la définition par les ouverts, et la définition par les voisinages d'un point
% 	\begin{description}
% 		\item [\emph{Par les ouverts}] : couple (E, T), où E est un ensemble et T une topologie sur E, à savoir un ensemble de parties de E — que l'on appelle les ouverts de (E, T);
% 		\item [\emph{Par les voisinages}] : Un espace topologique est un couple $(E,\mathcal {V})$, où $E$ est un ensemble et $\mathcal {V}$ une application de $E$ vers l'ensemble $P(P(E))$ obéissant aux cinq conditions ci-après, dans lesquelles les éléments de $\mathcal {V}(a)$, pour $a \in E$, sont appelés « voisinages de $a$ ». Pour tout point $a$ de $E$ :
% 		\begin{itemize}
% 			\item tout sur-ensemble d'un voisinage de $a$ est lui-même voisinage de $a$ ;
% 			\item l'intersection de deux voisinages de $a$ est elle-même un voisinage de $a$ ;
% 			\item $E$ est un voisinage de $a$ ;
% 			\item tout voisinage de $a$ contient $a$ ;
% 			\item pour tout voisinage $V$ de $a$, il existe un voisinage $W$ de $a$ tel que $V$ soit voisinage de chacun des points de $W$.
% 		\end{itemize}
% 	\end{description}},
% 	plural={espaces topologiques}
% }

% \newglossaryentry{partie}
% {
% 	name={partie},
% 	description={(mathématiques) Sous ensemble dont tous les éléments se trouvent dans un ensemble l'incluant}, 
% 	plural={parties}
% }

% \newglossaryentry{degree}
% {
% 	name={degr\'e},
% 	description={(mathématiques)(théorie des graphes) nombre de liens (arêtes ou arcs) reliant un sommet, avec les boucles comptées deux fois}, 
% 	plural={degr\'es},
% 	sort={degree}
% }

% \newglossaryentry{powerLaw}
% {
% 	name={loi de puissance},
% 	description={[\textit{power law}] (mathématiques) relation entre deux quantités x et y qui peut s'écrire de la façon suivante : \(y=ax^{k}\)}, 
% 	plural={lois de puissance}
% }

% \newglossaryentry{scaleFreeNet}
% {
% 	name={r\'eseau invariant d'échelle},
% 	description={[\textit{scale free network}] (mathématiques) réseau dont les degrés suivent une loi de puissance. C'est à dire un réseau où la proportion de nœuds de degré $k$ est proportionnelle à $k^{-\gamma}$ pour $k$ grand, où $\gamma$ est un paramètre (situé entre 2 et 3 pour la plupart des applications car en deçà ce n'est plus une loi de puissance, et au delà par nature cela réduirait ou augmenterait beaucoup trop l'espace de définition, ce qui devient dur à modéliser sur un plot)}, 
% 	plural={r\'eseaux invariants d'échelle},
% 	sort={reeseau invariant d'eechelle}
% }

% \newglossaryentry{topologie}
% {
% 	name={topologie},
% 	description={(mathématiques) Une topologie sur un ensemble $E$ est une famille $O \subset P(E)$ de parties de $E$ vérifiant 3 conditions : 
% 	\begin{itemize}
% 		\item $\varnothing$ et $E$ sont des éléments de $O$;
% 		\item Toute réunion d'éléments de $O$ est un élément de $O$;
% 		\item Toute intersection finie d'éléments de $O$ est un élément de $O$.
% 	\end{itemize}
% 	Les éléments de $O$ sont appelés les ouverts de la topologie. Le couple $(E, O)$ est appelé un espace topologique.\\\emph{Exemple} : L'ensemble $E=\{1,2,3\}$ peut être muni de 29 topologies
% 	\begin{itemize}
% 		\item $O_1 = \{\varnothing, E\}$
% 		\item $O_2 = \{\varnothing, \{1\}, E\}$
% 		\item $O_3 = \{\varnothing, \{1\}, \{1,2\}, E\}$
% 		\item $O_4 = \{\varnothing, \{1\}, \{1,2\}, \{1,3\}, E\}$
% 		\item $O_5 = \{\varnothing, \{2\}, \{1,2\}, \{2,3\}, E\}$
% 		\item $O_6 = \{\varnothing, \{1\}, \{2\}, \{1,2\}, \{1,3\}, E\}$
% 		\item \dots
% 	\end{itemize}},
% 	plural={topologies}
% }

% \newglossaryentry{ensemble}
% {
% 	name={ensemble},
% 	description={(mathématiques) collection d’objets (les éléments de l'ensemble) qui peuvent être compris comme un tout. Se note $E$}, 
% 	plural={ensembles}
% }

% \newglossaryentry{ensembleDesPart}
% {
% 	name={ensemble des parties},
% 	description={(mathématiques) Ensemble des sous-ensembles de cet ensemble. Se note $P(E)$.\\\emph{Exemple} : Soit $E$ un ensemble tel que $E=\{1,2,3\}$, alors $P(E)=\{\{a\},\{b\},\{c\},\{a,b\},\{a,c\},\{b,c\} ,\{a,b,c\},\{\varnothing\}\}$}, 
% 	plural={ensembles des parties}
% }

% \newglossaryentry{ouvert}
% {
% 	name={ouvert},
% 	description={(mathématiques)(topologie) aussi appelé ensemble ouvert ou une partie ouverte, un ouvert est un sous-ensemble d'un espace topologique qui ne contient aucun point de sa frontière\\\emph{Exemple} : Les points $(x, y)$ qui satisfont à l'équation $x^2 + y^2 = r^2$, soit un cercle, forment la frontière d'un espace topologique. Les points tels que $x^2 + y^2 < r^2$ forment un ensemble ouvert. L'union de cette frontière et de cet ensemble ouvert forme un ensemble fermé}, 
% 	plural={ouverts}
% }

% \newglossaryentry{recouvrement}
% {
% 	name={recouvrement},
% 	description={[\textit{overlap}] (mathématiques) Le recouvrement d'un ensemble $E$ est une famille $(X_i)_{i\in I}$ d'ensembles dont l'union contient $E$ (c'est à dire que tout élément de $E$ appartient au moins à l'un des $X_i$)},
% 	plural={recouvrements}
% }

% \newglossaryentry{QTL}
% {
% 	name={locus de caract\`eres quantitatifs},
% 	description={[\textit{quantitative trait locus}] (biologie) région plus ou moins grande d'ADN qui est étroitement associée à un caractère quantitatif, c'est-à-dire une région chromosomique où sont localisés un ou plusieurs gènes à l'origine du caractère en question}, 
% 	plural={loci de caract\`eres quantitatifs}
% }

% \newglossaryentry{autosome}
% {
% 	name={autosome},
% 	description={(biologie) chromosome non sexuel}, 
% 	plural={autosomes}
% }

% \newglossaryentry{gonosome}
% {
% 	name={gonosome},
% 	description={(biologie) chromosome sexuel}, 
% 	plural={gonosomes}
% }

% \newglossaryentry{ADNgenomiq}
% {
% 	name={ADN g\'enomique},
% 	description={(biologie) ADN se situant dans le nucléole, lui même se situant dans le noyau}, 
% 	plural={ADNs g\'enomiques}
% }

% \newglossaryentry{hubGene}
% {
% 	name={hub gene},
% 	description={(biostatistique) gène d'un module qui tend à avoir une très forte connectivité intra-modulaire. Il est généralement très fortement corrélé avec l'eigengene du même module}, 
% 	plural={hub genes}
% }

% \newglossaryentry{eigengene}
% {
% 	name={eigengene},
% 	description={(biostatistiques) L'eigengene d'un module de gènes est la première composante principale de ce module. Il peut être considéré comme le représentant des profils d'expression des gènes présents dans le module}, 
% 	plural={eigengene}
% }

% \newglossaryentry{geneSignif}
% {
% 	name={importance significative d'un g\`ene},
% 	description={[\textit{gene significance}] (bio-statistiques) valeur absolue de la corrélation entre un gène et un trait phénotypique }, 
% 	plural={importance significative de g\`enes}
% }

% \newglossaryentry{geneMember}
% {
% 	name={appartenance d'un g\`ene},
% 	description={[\textit{gene membership}] (biostatistiques) corrélation entre l'eigengene d'un module et le profil d'expression d'un gène}, 
% 	plural={appartenance de g\`enes}
% }

