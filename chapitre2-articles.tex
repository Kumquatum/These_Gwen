\setcounter{chapter}{3}
\chapter*{Chapitre 2 - À définir}
\phantomsection\addcontentsline{toc}{chapter}{Chapitre 2 - À définir} 
%%\chapter{Titre du chapitre}     % numéroté

Idées :
\begin{itemize}
    \item Rajout d'information protéique pour "consolider" le réseau ?
    \item Rajout d'une méthode de réseau consensus à GWENA ?
    \item Differential co-expression avec un autre tissu sur lequel on peut avoir des jeux de données jeune/vieux pour récup les genes communs au vieillissement, ceux specifique à un tissu ou l'autres, et ceu combinant le vieillissement specifique au tissu ?
    \item Une étude de l'impacte de la filtration sur l'état du réseau final ? Il n'y a rien de systemique dans la littérature, personne qui en fasse une review. Parsana et al. (alexis battle team) a fait ça en 2019 pour tester la PC-correction uniquement. Ils ont simulé des données scale free + des données scale free mimiquant GTEx et essayé de déterminer l'impact en calculant un FDR
\end{itemize}

\section{test bis}

\bibliographystyle{}              % style de la bibliographie
\bibliography{}                   % production de la bibliographie
