% \setcounter{chapter}{3}
\chapter{Chapitre 2 - L'analyse multidimensionnelle par réseau de co-expression de gènes}
% \chapter*{Chapitre 2 - L'analyse multidimensionnelle par réseau de co-expression de gènes}
% \phantomsection\addcontentsline{toc}{chapter}{Chapitre 2 - À définir} 
%%\chapter{Titre du chapitre}     % numéroté
\label{chapter:multidim}

% \section{Résumé}
\todo[inline]{Causer avec MP de l'utilisation de "mécanisme" ou "fonction" (biologique) de façon interchangeable : bonne ou mauvaise idée ?}

La variation de la distribution de la co-expression des gènes associée au vieillissement est une propriété qu'on a pu observer dans le muscle lors du chapitre précédent. Grâce à elle, on a pu isoler dans le chapitre précédent des gènes dont la fonction biologique est confirmée comme liée au vieillissement du muscle. On peut alors considérer l'étude de la topologie des réseaux de co-expression de gènes comme une méthode efficace pour la détection de gènes biomarqueurs du vieillissement dans un tissu. 

Cependant le vieillissement est un phénomène qui ne se limite pas au muscle dans un organisme et touche bien d'autres tissus en parallèle. Les manifestations diffèrent toutefois d'un tissu à l'autre.
% du fait de la variété dans la composition cellulaire des tissus. % <= pas que donc pas assez bien (ex: photo vieillissement)
Dans chacun d'eux, l'analyse d'expression différentielle ainsi que d'autres méthodes d'analyse de gènes dites discrètes \cite{Barabasi2004} ont permis de relever nombre de biomarqueurs individuels du vieillissement. 


Il se manifeste dans chacun sous des formes  différente avec pour certains
Une difficulté supplémentaire se présente alors lors de l'isolement de biomarqueurs : les changements biologiques entraînés par l'altération de ces biomarqueurs lors du vieillissement sont-ils 
entraînés par des mécanismes distincts au sain du tissu considéré ou bien sont-ils communs entre plusieurs changements et/ou tissus ?
% un phénomène local au tissu ou bien plus général avec une incidence dans le tissu considéré ?
% Sont-ils la cause ou la conséquence du vieillissement. 
Une même modification avec le temps telle que le raccourcissement des télomères (abordé en X.X.X\todo{ref intro vieillissement}) peut ainsi donner des affections variable selon le tissu : fibroses pulmonaires, des anémies aplasique, des dyskératoses congénitale, etc. \cite{Armanios2012} (Table \ref{table:tissu_telomere_effet}). 

\begin{table}[h]
\resizebox{\textwidth}{!}{
\begin{tabular}{m{0.2\textwidth}m{0.2\textwidth}m{0.6\textwidth}}
\textbf{Type de tissu}                                                                    & \textbf{Nom du tissu}                                            & \textbf{\begin{tabular}[c]{@{}l@{}}Manifestations pathologiques chez l'humain \\ avec syndromes télomériques\end{tabular}} \\ \hline
\multirow{4}{*}{\begin{tabular}[c]{@{}l@{}}Tissus à fort\\ renouvellement\end{tabular}}   & Peau                                                             & • Blanchissement prématuré des cheveux                                                                                     \\
                                                                                          & \begin{tabular}[c]{@{}l@{}}Moelle \\ osseuse\end{tabular}        & • Anémie aplastique                                                                                                        \\
                                                                                          & Immune                                                           & \begin{tabular}[c]{@{}l@{}}• Infections opportuniste\\ • Immunodéfiscience des cellules B, T et NK\end{tabular}            \\
                                                                                          & \begin{tabular}[c]{@{}l@{}}Épithélium \\ intestinal\end{tabular} & \begin{tabular}[c]{@{}l@{}}• Entérocolite\\ • Émoussement villositaire\end{tabular}                                        \\
\multirow{3}{*}{\begin{tabular}[c]{@{}l@{}}Tissus à faible\\ renouvellement\end{tabular}} & Poumon                                                           & \begin{tabular}[c]{@{}l@{}}• Emphysème prématuré\\ • Fibrose pulmonaire idiopathique\end{tabular}                          \\
                                                                                          & Foie                                                             & • Fibrose-cirrhose du foie cryptogénétique                                                                                 \\
                                                                                          & Os                                                               & \begin{tabular}[c]{@{}l@{}}• Ostéoporose\\ • Nécrose avasculaire\end{tabular}                                              \\
Cancer                                                                                    & Multi-tissus                                                     & \begin{tabular}[c]{@{}l@{}}• Cancers épithéliaux\\ • Hémopathies malignes\end{tabular}                                    
\end{tabular}
}
\caption{Phénotypes de maladies spécifiques aux organes chez les humains ayant des télomères courts. Modifié d'après la Table 2 de Armanios 2012, \textit{The telomere syndromes} \cite{Armanios2012}}
\label{table:tissu_telomere_effet}
\end{table}



Des études précédentes tendent à démontrer l'existence de bases communes au vieillissement \cite{DeMagalhaes2009a}. On relève ainsi des signatures d'expression de gènes communes à plusieurs tissus dans leur état "âgé". Parmi les fonctions détectées comme sur-exprimées par l'expression différentielle, on retrouve notamment des composantes de la réponse immunitaire/inflammatoire et de la dégradation lysosomale. À l'opposé, dans les fonctions détectées comme sous-exprimées, on retrouve des fonctions associées à l'encodage de protéines mitochondriales ainsi que des gènes responsables de la production de différents types de collagène.

% ### REDO


% ### END


Cependant, la diversité des fonctions impliquées complexifie la compréhension des relations qui les unirait, et ce malgré l'interconnectivité des caractéristiques majeures du vieillissement définies par Lopèz et al. \cite{Lopez-Otin2013}. Le ciblage de gènes uniques permis par l'analyse d'expression différentiele apporte certes une connaissance sur les fonctions impliquées dans le vieillisseme

Cependant comme on a pu l'exposer auparavant \todo{raccrocher les wagons avec l'intro quand elle sera écrite pour parler de l'avantage à utiliser des réseaux par rapport à de la simple DE}, le ciblage de gènes uniques comme le fait l'expression différentielle ne permet pas de comprendre au mieux les relations qui les unis. Ainsi il est plus compliqué d'estimer si leur sélection est issue de mécanismes du vieillissements distincts ou communs. Des études ciblées sur chacun des gènes apportent des réponses comme le rapporte la base de donnée CellAge \cite{Avelar2019} ou encore AGEMAP chez la souris \cite{Zahn2007} mais restent toutefois trop peu nombreuses pour couvrir la totalité des gènes impliqués dans le vieillissement ainsi que leurs interactions. 

Dans ce chapitre, on s'attachera donc à déterminer si l'analyse par réseaux de co-expression de gènes permet potentiellement de répondre à ce besoin \todo{Peut etre reformuler pour que ça fasse plus hypothès/problématique}. On commencera traiter les données issues de différents tissus tout en expliquant les précautions prises, on poursuivra par l'exploration générale des modules de gènes obtenus, et enfin on s'attardera sur les apports du croisement de données entre modules modérément et/ou non préservés. \todo{Reverifier plus tard si ce plan tient toujours, notamment si on a pas rajouté un aspect etude de la déconnexion}

% Objectif principal : 
% Étudier la variation de la déconnexion entre les tranches d'âge extrêmes 
% dans de multiples tissus

% Objectifs secondaires :

% * Vérifier le nombre d'échantillons par tranche d'âge de 10 ans
% * Identifier les tissus d'intérêt pour explorer la déconnexion.
% * Détecter les modules avec chute de connectivité.
% * Déterminer les gènes communs et les gènes spécifiques, et leurs interactions.
% * Explorer les modèles de déconnexion entre les tissus.

\section{Données et pré-traitement}

\subsection{Description des données}

Comme dans le chapitre \ref{chapter:gwena}, on a utilisé les données issues de l'étude GTEx \cite{Ardlie2015}. Pour aller plus en détail que ce que ne le permet le format d'article du chapitre \ref{chapter:gwena}, ces données sont le regroupement d'échantillons prélevés sur 54 tissus (+1 tissu qui est en fait un assemblage de lignées cellulaires dérivées de patients atteints de leucémie myéloïde aiguë\cite{Way2020}) et 948 donneurs dans sa dernière version, la v8 (Table \ref{table:gtex_sample_tissues_donnor}. Cette variété de tissus vise à être le plus représentatif possible (au vu du coup de prélèvement et d'analyse de chaque échantillon) des différents tissus chez l'humain. Les biopsies sont effectuées sur des donneurs décédés (avec leur accord préalable, l'accord d'un proche, ou l'accord du représentant légal) dans 4 centres de collecte, puis analysées sur place ou transférées selon le tissu avec réfrigération durant le transport. Ces échantillons sont évalués sur plusieurs critères pour jauger leur admissibilité : des critères cliniques (absence de contamination au VIH, absence de chimiothérapie dans les 2 ans, absence de transfusion sanguine dans les 48h, etc.), des critères anatomopathologiques (absence de tissu cancéreux, absence de pathologie tissulaire, etc.) et des critères analytiques (quantité de tissu prélevé suffisante, quantité d'ARN extrait final supérieur à 500ng d'ARN total, nombre d'intégrité d'ARN ou RIN supérieur à 5.7) \cite{Carithers2015}. Tout échantillon non conforme est exclu de la cohorte.

Sur la majorité intégralité des échantillons ont été effectué des séquençages de génome (\textit{Whole Genome Sequencing}, WGS), des séquençages d'exome complet (\textit{Whole Exome Sequencing}, WES), des séquençage de transcriptome (aussi appelé séquençage d'ARN, RNA-Seq), ainsi que des des images de coupes histologiques colorées. Des données reformatées sont également mises à disposition telle que les locus de caractères quantitatifs (\textit{quantitative trait loci}, QTL) et l'expression de gène qui est ce que l'on va utiliser ici. Ces données sont disponibles sur le site du consortium GTEx (\url{https://gtexportal.org}) accompagnées d'information sur le phénotype des échantillons. En raison du fort potentiel d'identification des donneurs, le phénotype donné publiquement est partiel, et le phénotype complet est disponible sur demande auprès de dbGaP après soumission d'un dossier de projet à renouveler chaque année (Annexe \ref{annexe:dbgap}).

\begin{table}[h]
\centering
\begin{tabular}{llll}
\textbf{GTEx V8} & \textbf{\# Tissus} & \textbf{\# Donneurs} & \textbf{\# Échantillons} \\ \hline
Total            & 54                 & 948                  & 17382                    \\
Avec Genotype    & 54                 & 838                  & 15253                    \\
Avec eQTL        & 49                 & 838                  & 15201                   
\end{tabular}
\caption{Nombre de tissus, donneurs et échantillons disponibles selon les données}
\label{table:gtex_sample_tissues_donnor}
\end{table}

Par ailleurs, tous les donneurs ne peuvent pas être prélevés pour l'ensemble des 54 tissus et ce sont en moyenne 23,4 tissus qui sont prélevés (v8). Certains tissus ont par ailleurs été priorisés lors des biopsies : tissu adipeux (sous-cutané), artère tibiale, coeur (ventricule gauche), poumon, muscle (squelettique), nerf tibial, peau (exposée au soleil), thyroïde, sang complet. Parmi les différents tissus biopsiés, il est à noter que certains sont issus d'un même organe et on a ainsi 31 organes biopsiés pour 54 tissus biopsiés en tout (Figure \todo{faire une figure avec cette info}).

\todo[inline]{Un Sankey diagramme des tissus avec col 1 = SMTS, col 2 = SMTSD, col 3 = tranche d'age à 10 ans}


\subsection{Sélection des tissus et filtre des échantillons}

Parmi les 54 tissus disponibles, tous n'étaient pas adaptés à l'étude globale du vieillissement chez l'humain. Ainsi un premier filtre a retiré tous les tissus liés à un seul sexe : Trompes de Fallope, Col de l'utérus, Utérus, Vagin, Sein, Ovaire, Prostate, Testicule. Un second filtre a été de retirer les échantillons de lignées cellulaires car non représentatifs du vieillissement biologique : Cellules - Lymphocytes transformés par EBV, Cellules - Fibroblastes cultivés, Cellules - Lignée cellulaire de leucémie (CML). Enfin, l'expression des tissus tumoraux a montré, dans des études préalables, des modèles d'expression génétique différents de ceux non tumoraux \todo{REF}. Par conséquent, ces tissus sont supprimées ici ainsi que les échantillons dont le statut cancéreux 
était inconnu. 

\todo{Faire une figure avec l'évolution de la disparition du nombre de tissus}

Le vieillissement est un phénomène dont les altérations moléculaires sont linéaires avec le temps, dont les dommages cellulaires sont super-linéaires \cite{Todhunter2018}, et où la mortalité associée augmente de façon exponentielle passé 20 ans \cite{Finch2016}. Afin de faciliter la détection de ces altérations grâce à l'analyse de co-expression différentielle, je me suis donc dans un premier temps concentrée sur une sélection de tranches d'âges très contrastées. Ainsi, les échantillons sélectionnés sont issus de donneurs entre 20 et 30 ans pour la tranche qu'on nommera "jeune", et entre 60 et 70 ans pour la tranche qu'on nommera "âgée". Les données de GTEx ne comportent toutefois pas un nombre d'échantillons similaire pour chacune de ces tranches du fait de la mortalité plus importante chez les personnes âgées que les personnes jeunes (principalement des décès par traumatisme chez les jeunes plutôt que par maladie chronique ou maladie liée à l'âge chez les âgés). 

Parmi les tissus précédemment sélectionnés, j'ai donc à nouveau restreint le nombre de tissus à ceux comportant au minimum 50 échantillons dans chacune des tranches d'âge : Vessie (20) et Rein - Médulla (3). Ce nombre d'échantillons permet d'assurer un bon compromis entre des réseaux de co-expression de gènes robustes (donc non sensibles à des valeurs aberrantes) et la perte de plus de tissus à étudier dans cette analyse multidimensionnelle qu'on souhaite effectuer \cite{Liesecke2019}.

En plus de ces sélections de tissus, seuls les échantillons répondant au critère d'inclusion dans la cohorte "GTEx Analysis Freeze" ont été retenus. Cette cohorte atteste que les échantillons n'étaient pas issus de donneurs ayant des liens de parenté ou de donneurs considérés comme des "anomalies" \todo{S'il y a une formule moins péjorative, je prends...} biologiques, par exemple des donneurs avec des duplications/déletions chromosomiques, affectés d'un syndrome tel que défini par la base de données OMIM \cite{Hamosh2005} (ce qui n'inclue pas les maladies liées au vieillissement donc), ou encore ayant effectué une chirurgie de transition de genre \todo{check ce que Maria répond pour la formulation correcte}. En finalité, ce sont 36\% des échantillons qui ont été retirés.


\subsection{Correction des facteurs confondants}

À l'instar de l'analyse d'expression différentielle, l'analyse de co-expression différentielle nécessite des données les moins biaisées possibles pour une construction de réseau de qualité. Ces biais (ou facteurs confondants) peuvent être tant techniques que biologiques et vont entraîner une augmentation de la variation de façon artificielle. Le risque est alors de détecter une différence et de l'assumer comme issue de l'expérience qu'on mène alors qu'elle est en fait due à ces biais. Dans le cas de la co-expression spécifiquement, ces bais peuvent entraîner des corrélations crédibles mais erronées entre certains gènes, altérant alors la construction du réseaux de co-expression et donc la détection des modules. Ainsi, il est essentiel de corriger ces facteurs confondants au préalable de l'utilisation de mon package GWENA /todo{REF} sur les données.

La complexité de la correction des facteurs confondant réside dans leur suppression sans pour autant supprimer le signal expérimental d'intérêt, ici les variations dans le transcriptome dûs à l'âge. Comme précisé en \ref{subsubsection:microarray_props_and_normalisation} et \ref{subsubsection:rnaseq_props_and_normalisation},
chaque technologie de séquençage a un premier ensemble de normalisations spécifique à elle. Dans le cadre de la construction de réseaux de co-expression de gènes, une attention particulière doit être portée en plus du fait des risques de fausse corrélations. Malgré le progrès de techniques ciblées sur des facteurs confondants connus comme le sexe, le poids, l'héritage génétique, celles-ci ne sont pas capables de corriger pour des facteurs peu explicites ou diffus comme la classe sociale, l'alimentation, la façon de manipuler des techniciens. La correction par composante principale est une méthode ayant montré de bons résultats (basé sur de la détection de pathways au sain de modules détectés) de corrections de facteurs confondant par rapport à l'utilisation de correction par regression multible, par le taux exonic, ou encore par le numéro d'intégrité d'ARN (\textit{RNA integrity number}, ou RIN). Cependant cette méthode va également corriger pour l'âge si utilisée telle quelle,


À celle-ci s'ajoute des étapes de normalisation communes comme la correction du contenu en GC, 


La normalisation des gènes doivent toutefois être adaptés dans le cadre de la construction de réseaux de  co-expression de gènes. 



%% NON :  là mélange de filtration et normalisation
% En effet, le signal lié à la topologie d'invariance d'échelle doit être préservé, chose que plusieurs méthodes ne conservent pas comme la sélection des gène différentiellement exprimés uniquement. 




% Pour ce faire, plusieurs méthode (dont beaucoup issues de l'analyse d'expression différentielle) existent 
% Pour supprimer les biais biologiques on retrouve des méthodes 


\subsection{Construction des réseaux par tissu et détection des modules}



\begin{itemize}
    \item Tissu sanguin non exploitable
    \item 10 tissus exploitables
    \item 
    
\end{itemize}

\begin{table}[h]
\resizebox{\textwidth}{!}{
\begin{tabular}{llllll}
\textbf{}            & \textbf{Preserved} & \textbf{Moderately preserved} & \textbf{Unpreserved} & \textbf{Inconclusive} & \textbf{Total} \\ \hline
Adipeux sous-cutané & 12                 & 4                             & 0                    & 2                     & 18             \\
Artère tibiale        & 13                 & 8                             & 1                    & 1                     & 23             \\
Muqueuse œusophagienne     & 3                  & 8                             & 1                    & 0                     & 12             \\
Muscle œusophagien & 9                  & 6                             & 0                    & 6                     & 21             \\
Muscle squelettique      & 14                 & 13                            & 5                    & 3                     & 35             \\
Nerf tibial         & 35                 & 9                             & 1                    & 4                     & 49             \\
Peau non exposée au soleil & 36                 & 4                             & 0                    & 0                     & 40             \\
Peau exposée au soleil     & 10                 & 8                             & 2                    & 1                     & 21             \\
Thyroïde              & 12                 & 8                             & 1                    & 4                     & 25            
\end{tabular}
}
\caption{Nombre de modules par tissu et leur répartition selon leur statut de préservation.}
\label{table:modules_status_all_tissues}
\end{table}



% Idées :
% \begin{itemize}
%     \item Rajout d'information protéique pour "consolider" le réseau ?
%     \item Rajout d'une méthode de réseau consensus à GWENA ?
%     \item Differential co-expression avec un autre tissu sur lequel on peut avoir des jeux de données jeune/vieux pour récup les genes communs au vieillissement, ceux specifique à un tissu ou l'autres, et ceu combinant le vieillissement specifique au tissu ?
%     \item Une étude de l'impacte de la filtration sur l'état du réseau final ? Il n'y a rien de systemique dans la littérature, personne qui en fasse une review. Parsana et al. (alexis battle team) a fait ça en 2019 pour tester la PC-correction uniquement. Ils ont simulé des données scale free + des données scale free mimiquant GTEx et essayé de déterminer l'impact en calculant un FDR
% \end{itemize}










Tous les donneurs n'ont toutefois pas pu être prélevés pour l'ensemble des 54 tissus, entraînant une première disparité dans les données.

